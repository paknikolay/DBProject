\documentclass{article}

\usepackage[T1,T2A]{fontenc}
\usepackage[utf8]{inputenc}
\usepackage[russian]{babel}
\usepackage{ dsfont }
\usepackage{mathtools}
\usepackage{ gensymb }
\usepackage{ amssymb }
\usepackage[left=1cm,right=1cm,
top=1cm,bottom=2cm,bindingoffset=0cm]{geometry}
\usepackage{fancyhdr}
\begin{document}
\LARGE
Проект. Пак Николай, Проценко Никита:
Данный проект предназначен для: \\
\begin{enumerate}
	\item Для просмотра расписания на любой день недели (дата, время, корпус, кабинет, предмет, преподаватель)
	\item Для нахождения свободного времени группы лиц
	\item Поиск контактной информации по конкретному человеку (преподаватель, ученик)
	\item Общая информация по ВУЗу(факультету, кафедре, предметах)
	\item Поиск общих предметов с несколькоми людьми
	\item Расписание на завтра(без лишних выборок в отличие от 1 пункта)
	\item Поддерживение пользователей
	\item Графическое приложение(надеюсь)
	\item Может еще что нибудь \ldots )
\end{enumerate}
	
\end{document}